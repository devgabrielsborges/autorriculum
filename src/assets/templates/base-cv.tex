%%%%%%%%%%%%%%%%%%%%%%%%%%%%%%%%%%%%%%%%%
% Template de Currículo Profissional
%
% Autor: Seu Nome
% Baseado em pesquisas sobre currículos eficazes e ATS-friendly.
%
% Este template é projetado para ser:
% - Limpo e profissional
% - Facilmente legível por recrutadores e sistemas ATS
% - Simples de personalizar
%%%%%%%%%%%%%%%%%%%%%%%%%%%%%%%%%%%%%%%%%

%----------------------------------------------------------------------------------------
%	CONFIGURAÇÕES DO DOCUMENTO
%----------------------------------------------------------------------------------------

\documentclass[a4paper,11pt]{article}

% --- PACOTES ESSENCIAIS ---
\usepackage[T1]{fontenc}
\usepackage[utf8]{inputenc}
\usepackage{lmodern} % Fonte moderna
\usepackage{geometry} % Para ajustar as margens
\usepackage{hyperref} % Para links clicáveis (email, websites)
\usepackage{fontawesome5} % Para ícones (LinkedIn, GitHub, etc.)
\usepackage{titlesec}   % Para personalizar títulos de secção
\usepackage{enumitem}   % Para personalizar listas

% --- CONFIGURAÇÃO DAS MARGENS ---
% Margens generosas para uma boa legibilidade
\geometry{
    a4paper,
    left=2cm,
    right=2cm,
    top=2cm,
    bottom=2cm
}

% --- CONFIGURAÇÃO DE CORES E LINKS ---
\usepackage{xcolor}
\definecolor{linkcolor}{rgb}{0.0, 0.42, 0.72} % Cor azul para links
\hypersetup{
    colorlinks=true,
    urlcolor=linkcolor,
    linkcolor=linkcolor,
}

% --- REMOVER NÚMERO DA PÁGINA ---
\pagestyle{empty}

% --- CONFIGURAÇÃO DOS TÍTULOS DAS SECÇÕES ---
% Define um estilo limpo para as secções, com uma linha por baixo
\titleformat{\section}{
  \vspace{8pt} % Espaço antes do título
  \scshape\Large % Letras maiúsculas pequenas, tamanho grande
}{}{0em}{}[\titlerule] % Linha horizontal após o título
\titlespacing{\section}{0pt}{10pt}{5pt} % Espaçamento: esquerda, antes, depois

% --- COMANDO PERSONALIZADO PARA ENTRADA DE EXPERIÊNCIA/EDUCAÇÃO ---
% Isso garante consistência na formatação
\newcommand{\resumeEntry}[4]{
  \vspace{4pt}
  \textbf{#1} \hfill \textit{#2} \\
  \textit{#3} \hfill \textit{#4} \\
}

%----------------------------------------------------------------------------------------

\begin{document}

%----------------------------------------------------------------------------------------
%	CABEÇALHO - SEUS DADOS
%----------------------------------------------------------------------------------------

\begin{center}
    {\Huge \scshape SEU NOME COMPLETO} % Seu nome em destaque
    \vspace{6pt}

    % --- Informações de Contato com Ícones ---
    % Use \faIcon{icon-name} para os ícones. Ex: github, linkedin, envelope
    \small
    \faIcon{map-marker-alt} Sua Cidade, Estado \quad%
    \href{mailto:seu.email@profissional.com}{\faIcon{envelope} seu.email@profissional.com} \quad%
    \href{https://www.linkedin.com/in/seu-perfil}{\faIcon{linkedin} /seu-perfil} \quad%
    \href{https://github.com/seu-usuario}{\faIcon{github} /seu-usuario}
\end{center}

%----------------------------------------------------------------------------------------
%	RESUMO PROFISSIONAL
%----------------------------------------------------------------------------------------

\section{Resumo Profissional}
\vspace{2pt}
% Um parágrafo curto (3-4 linhas) que resume sua experiência, suas principais competências e seu objetivo de carreira.
% Adapte este texto para cada vaga à qual se candidatar.
Desenvolvedor de Software com X anos de experiência, especializado em [Sua Especialidade, ex: desenvolvimento Back-End com Node.js]. Possuo forte conhecimento em [Tecnologia 1] e [Tecnologia 2], com histórico comprovado em [Sua maior conquista, ex: otimização de sistemas e redução de latência em 40%]. Busco uma oportunidade desafiadora para aplicar minhas habilidades em [Área de Interesse].

%----------------------------------------------------------------------------------------
%	COMPETÊNCIAS TÉCNICAS
%----------------------------------------------------------------------------------------

\section{Competências Técnicas}
\begin{itemize}[leftmargin=*, label={}]
    \item \textbf{Linguagens:} JavaScript, TypeScript, Python, Java, C
    \item \textbf{Back-End:} Node.js, Express.js, NestJS, Spring Boot
    \item \textbf{Front-End:} React, Angular, Vue.js, HTML5, CSS3
    \item \textbf{Bancos de Dados:} PostgreSQL, MongoDB, Redis, MySQL
    \item \textbf{Cloud \& DevOps:} AWS (EC2, S3, Lambda), Docker, Kubernetes, Jenkins, Git
    \item \textbf{Outras Ferramentas:} Metodologias Ágeis (Scrum, Kanban), Jest, Postman
\end{itemize}

%----------------------------------------------------------------------------------------
%	EXPERIÊNCIA PROFISSIONAL
%----------------------------------------------------------------------------------------

\section{Experiência Profissional}

\resumeEntry
  {Seu Cargo Mais Recente} % Cargo
  {Nome da Empresa} % Empresa
  {Cidade, Estado} % Localização
  {Mês, Ano – Presente} % Período

\begin{itemize}[leftmargin=*, topsep=2pt, itemsep=2pt]
    \item Descreva aqui sua principal conquista, usando verbos de ação e quantificando o resultado. Ex: Liderei o desenvolvimento de uma nova arquitetura de microsserviços, o que reduziu o tempo de deploy em 50\%.
    \item Outra conquista mensurável. Ex: Otimizei consultas em PostgreSQL e implementei cache com Redis, resultando em uma diminuição de 60\% no tempo de resposta de APIs críticas.
    \item Mais uma responsabilidade ou conquista. Ex: Desenvolvi e mantive pipelines de CI/CD com GitHub Actions, automatizando testes e deployments.
\end{itemize}

\resumeEntry
  {Seu Cargo Anterior} % Cargo
  {Nome da Empresa Anterior} % Empresa
  {Cidade, Estado} % Localização
  {Mês, Ano – Mês, Ano} % Período

\begin{itemize}[leftmargin=*, topsep=2pt, itemsep=2pt]
    \item Conquista ou responsabilidade. Ex: Desenvolvi e mantive APIs RESTful com Node.js e Express para um sistema de gestão de clientes (CRM).
    \item Conquista ou responsabilidade. Ex: Participei da migração de um sistema monolítico para uma arquitetura baseada em serviços, utilizando Docker.
    \item Conquista ou responsabilidade. Ex: Criei testes unitários e de integração com Jest, aumentando a cobertura de testes do módulo de 40\% para 85\%.
\end{itemize}

%----------------------------------------------------------------------------------------
%	PROJETOS
%----------------------------------------------------------------------------------------

\section{Projetos}

\resumeEntry
  {Nome do Projeto 1} % Nome do Projeto
  {\href{https://github.com/seu-usuario/seu-projeto}{github.com/seu-usuario/seu-projeto}} % Link
  {} % Localização (pode deixar em branco)
  {Ano} % Período

\begin{itemize}[leftmargin=*, topsep=2pt, itemsep=2pt]
    \item \textbf{Descrição:} Uma breve descrição do projeto, o problema que ele resolve e o seu papel.
    \item \textbf{Tecnologias:} Liste as tecnologias, frameworks e ferramentas que você utilizou.
\end{itemize}

\resumeEntry
  {Nome do Projeto 2} % Nome do Projeto
  {\href{https://github.com/seu-usuario/seu-projeto-2}{github.com/seu-usuario/seu-projeto-2}} % Link
  {} % Localização (pode deixar em branco)
  {Ano} % Período

\begin{itemize}[leftmargin=*, topsep=2pt, itemsep=2pt]
    \item \textbf{Descrição:} Uma breve descrição do projeto, o problema que ele resolve e o seu papel.
    \item \textbf{Tecnologias:} Liste as tecnologias, frameworks e ferramentas que você utilizou.
\end{itemize}

%----------------------------------------------------------------------------------------
%	FORMAÇÃO ACADÊMICA
%----------------------------------------------------------------------------------------

\section{Formação Acadêmica}

\resumeEntry
  {Seu Curso Superior} % Curso
  {Nome da Universidade} % Instituição
  {Cidade, Estado} % Localização
  {Ano de Conclusão ou Previsão} % Período

%----------------------------------------------------------------------------------------
%	IDIOMAS E CERTIFICAÇÕES
%----------------------------------------------------------------------------------------

\section{Idiomas e Certificações}
\begin{itemize}[leftmargin=*, label={}]
    \item \textbf{Português:} Nativo
    \item \textbf{Inglês:} Avançado / Fluente
    \item \textbf{Nome da Certificação 1} - Instituição Certificadora (Ano)
    \item \textbf{Nome da Certificação 2} - Instituição Certificadora (Ano)
\end{itemize}

\end{document}
